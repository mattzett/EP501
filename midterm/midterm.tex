\documentclass{article}
\usepackage{graphicx}
\usepackage{fullpage}
\usepackage{hyperref}
\usepackage{amsmath}
\usepackage{amssymb}
%\usepackage{draftwatermark}

%\SetWatermarkText{DRAFT}
%\SetWatermarkScale{3}
%\SetWatermarkLightness{0.5}

\begin{document}

\title{EP 501 Midterm exam:  Chapters 1,3-5}

\maketitle

~\\
\textbf{Instructions:}  
\begin{itemize}
  \item  Answer all questions
  \item  You may use the course textbook during this exam
  \item  You may log into your computer and use Matlab.
  \item  You may use \emph{your own} course notes for this exam
  \item  You may use my course notes.  
  \item  You may access the course Canvas site during this exam:  \url{https://erau.instructure.com/courses/101937}
  \item  You may visit the course repository:  \url{https://github.com/mattzett/EP501}  
  \item  You \emph{may not} use an internet browser to access search capabilities and internet references.  
\end{itemize}


\pagebreak

\begin{enumerate}
  \item Use the Taylor Series method (\S 5.4 in the textbook) to develop derivatives for a \emph{nonuniform} (unequally spaced) grid consisting of the three points $x_{i-1},x_i,x_{i+1}$ and function values at those points $f_{i-1},f_i,f_{i+1}$ defined by:
    \begin{eqnarray}
      x_{i-1} &=& x_i - \Delta x_b \\
      x_{i+1} &=& x_i + \Delta x_f \\
      \left( \Delta x_b \right. &\ne& \left. \Delta x_f \right) \\
      f_{i-1} &=& f(x_i - \Delta x_b) \\ 
      f_{i+1} &=& f(x_i + \Delta x_f)     
    \end{eqnarray}
    \begin{itemize}
      \item[(a)]  Develop a \emph{centered} difference approximation for the first derivative with respect to $x$ at the $i^{th}$ grid point, i.e. derive an approximation for:
      \begin{eqnarray}
      f'(x_i) = \left[ \frac{d f}{d x} \right]_i
      \end{eqnarray}    
    \item[(b)]  Show that the truncation error for your finite difference formula is $\mathcal{O}(\Delta x_f - \Delta x_b)$
    \item[(c)]  Obtain an approximate second derivative:
      \begin{eqnarray}
        f''(x_i) = \left[ \frac{d^2 f}{d x^2} \right]_i
      \end{eqnarray}        
      by iteratively applying your first derivative formula derived from part (a), e.g.:
      \begin{eqnarray}
        \left[ \frac{d^2 f}{d x^2} \right]_i \approx 
        \frac{
        \left[ \frac{d f}{d x} \right]_{i+1/2} - \left[ \frac{d f}{d x} \right]_{i-1/2}
        }{x_{i+1/2}-x_{i-1/2}}
      \end{eqnarray} 
    \end{itemize}
    
    \pagebreak
    
    (This page intentionally left blank for additional space for question 1)
    ~\\
    
    \pagebreak
    
    \item  Derive and explain matrix condition numbers.  
    \begin{itemize}
      \item[(a)]Use a method similar to that presented in \S 1.6.3.2 to show that the condition number relates variations in the solution vector $\underline{b}$ to variations in the matrix $\underline{\underline{A}}$ via the formula:
      \begin{equation}
      \frac{\left\| \delta \underline{b} \right\|}{ \left\| \underline{b} \right\| } \le   \mathcal{C} (\underline{\underline{A}}) \frac{ \left\| \delta \underline{\underline{A}}   \right\| }{\left\| \underline{\underline{A}} \right\|}
      \end{equation}
      where $\mathcal{C} (\underline{\underline{A}})$ is the condition number of the matrix $\underline{\underline{A}}$.      
    \item[(b)] Provide a brief (several sentence) explanation of the meaning of the condition number $\mathcal{C}(\underline{\underline{A}})$.  Include a discussion of what large and small condition numbers mean and why they are undesirable.  
    \end{itemize}
    
    \pagebreak
    
    \item \emph{Trilinear} interpolation involves approximation of an underlying three dimensional function using a polynomial of the form:  
    \begin{equation}
      f(x,y,z) \approx a_1 + a_2 x + a_3 y + a_4 z + a_5 x y + a_6 y z + a_7 x z + a_8 x y z
    \end{equation}
    for the region $x_i \le x \le x_{i+1}, y_j \le x \le y_{j+1}, z_k \le x \le z_{k+1}$.  Set up a system of equations that can be solved for the coefficients $\underline{a} \equiv a_\ell$ using the value of the function $f$ at the eight points defining the vertices of this cube-shaped region, shown in the diagram below.  Express your system of equations in matrix form.  \\~\\~\\
    %\begin{figure}{h*}
      \includegraphics[width=0.5\textwidth]{diagram_trilinear-crop.pdf}
    %\end{figure}
    
    \pagebreak
    
   \item Suppose we wish to perform a least squares fit (\S 4.10) to a set of measurements $y_i$ sampled at independent variable locations $x_i$ using the functional form:  
    \begin{equation}
      y(x) = a x^2 + b x^5
    \end{equation}
    Derive a system of equations can can be solved to determine the coefficients $a,b$.  Express your system in matrix form.  
    
\end{enumerate}

\end{document}
