\documentclass{article}
\usepackage{graphicx}
\usepackage{fullpage}
\usepackage{hyperref}
\usepackage{amsmath}
\usepackage{amssymb}
\usepackage{draftwatermark}

\SetWatermarkText{DRAFT}
\SetWatermarkScale{3}
\SetWatermarkLightness{0.5}

\begin{document}

\title{EP 501 Homework 4:  Differentiation, Integration, and Multidimensional functions}

\maketitle

~\\
\textbf{Instructions:}  
\begin{itemize}
  \item Submit all source code and publish Matlab results in .pdf form via Canvas.  Please zip all contents of your solution into single file and then submit in a single zip file.    
  \item Discussing the assignment with others is fine, but you must not copy anyone's code.  
  \item I must be able to run your code and produce all results by executing a single top-level Matlab script, e.g. \textsf{assignment1.m} or similar.  
  \item You may use any of the example codes from our course repository:  \url{https://github.com/mattzett/EP501/}.
  \item Do not copy verbatim any other codes (i.e. any source codes other than from our course repository).  You may use other examples as a reference but you must write you own programs (except for those I give you).  

\end{itemize}
~\\~\\~\\
\textbf{Purpose of this assignment:}  
\begin{itemize}
  \item Deal with numerical differentiation to solve complex problems.  
  \item Develop good coding and documentation practices, such that your programs are easily understood by others.  
  \item Exercise good judgement in numerical problem setup.
  \item Demonstrate higher reasoning in synthesize a problem and devising a basic algorithm to solve it.  
\end{itemize}

\pagebreak

\begin{enumerate}
  \item  Vector derivatives and multidimensional plotting:  
  \begin{itemize}  
    \item[(a)] Plot the two components of the vector magnetic field defined by the piecewise function:
    \begin{equation}
      \mathbf{B}(x,y)= \left\{
      \begin{array}{c}
      \frac{\mu_0 I}{2 \pi a^2} \sqrt{x^2+y^2} \left( - \frac{y}{\sqrt{x^2+y^2}}\hat{\mathbf{e}}_x + \frac{x}{\sqrt{x^2+y^2}} \hat{\mathbf{e}}_y \right) \qquad \left( \sqrt{x^2+y^2} < a \right)
      \\
      \frac{\mu_0 I}{2 \pi \sqrt{x^2+y^2}} \left( - \frac{y}{\sqrt{x^2+y^2}} \hat{\mathbf{e}}_x + \frac{x}{\sqrt{x^2+y^2}} \hat{\mathbf{e}}_y \right) \qquad \left( \sqrt{x^2+y^2} \ge a \right)
      \end{array}
      \right.      
    \end{equation}
    Use an image plot (\texttt{imagesc}, \texttt{pcolor}) for each component.  Make sure you add a colorbar and axis labels to your plot.  You will need to define a range and resolution in $x$ and $y$, and create a meshgrid from that.  Be sure to use a resolution fine enough to resolve important variations in this function.  
    \item[(b)]  Make a quiver plot of the magnetic field; add labels, etc.
    \item[(c)]  Write a function to compute the curl of a vector field, i.e. $\nabla \times \mathbf{U}$.  Compute the three components of the numerical curl of this field and plot using \texttt{imagesc}, or \texttt{pcolor}
    \item[(d)]  Compute the curl analytically and plot this alongside your numerical approximation.  Demonstrate that they are suitably similar.  
    \item[(e)]  Compute and plot the scalar field:
    \begin{equation}
    \Phi(x,y)=\frac{q}{4 \pi \epsilon_0} \frac{e^{-\alpha \sqrt{x^2+y^2} }}{\sqrt{x^2+y^2}} \left(1+\frac{\alpha \sqrt{x^2+y^2}} {2} \right)
    \end{equation}
    Make sure you avoid having $(x,y)=(0,0)$ in your grid (the function is singular at this point).  This is probably best done by implementing a ``regulator'' that enforces some minimum value for $x$ and $y$.  Be sure to use a resolution fine enough to resolve important variations in this function.
    \item[(e)]  Write a function to compute the Laplacian of a scalar field, i.e. $\nabla^2 \Phi$.  Compute the numerical Laplacian of the function:
    
    
  \end{itemize}
  \item  Integration in multiple dimensions.  
  \begin{itemize}
    \item[(a)] Numerically compute the electrostatic energy per unit length, defined by the integral:  
    \begin{equation}
      W_E = - \iint \left( \epsilon_0 \nabla^2 \Phi \right) \Phi dy dx
    \end{equation}
    using a method that you have coded yourself.  
    \item[(b)]  Compute and plot the parametric path
    \begin{equation}
       \mathbf{r}(\phi) = \cos \phi ~ \hat{\mathbf{e}}_x + \sin \phi ~ \hat{\mathbf{e}}_y \qquad (0 \le \phi \le 2 \pi)
    \end{equation}    
    in the $x,y$ plane on the same axis as your magnetic field components (plot the path in each).  You will need to define a grid in $\phi$ to do this.
    \item[(c)]  Plot the two components of the magnetic field at the $x,y$ points along $\mathbf{r}$ and visually compare against your image plots of the magnetic field.  
    \item[(d)]  Numerically compute the tangent vector to the path $\mathbf{r}$ by performing the derivative:  $d \mathbf{r} / d \phi$.  Compare your numerical results against the analytical derivative (e.g. plot the two) and adjust your grid in $\phi$ such that you get visually acceptable results.  
    \item[(e)]  Numerically compute the magnetic field integrated around the path $\mathbf{r}$, i.e.:
    \begin{equation}
       \int_{\mathbf{r}} \mathbf{B} \cdot d \boldsymbol{\ell} 
    \end{equation}
    where the differential path length is given by:
    \begin{equation}
      d \boldsymbol{\ell} = \frac{d \mathbf{r}}{d \phi} d \phi 
    \end{equation}   
  \end{itemize}
\end{enumerate}

\end{document}
